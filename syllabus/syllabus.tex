% Intro Syllabus: Fall 2019

\documentclass[hidelinks]{article}

%-- Activate below to make default font sans serif --%
%\renewcommand{\familydefault}{\sfdefault}

%-- Set Margins --%
\usepackage[margin=2.5cm]{geometry}

%-- Package for text colouration --%
\usepackage[usenames,dvipsnames]{color} 

%-- Package for Links --%
\usepackage{hyperref} 

%-- Set paragraph spacing and indentation --%
\setlength{\parindent}{0pt}
\setlength{\parskip}{0.5cm}

%-- Package for controlling spacing of lists --%
\usepackage[shortlabels]{enumitem}

%-- Packages for making nice tables --%
\usepackage{booktabs}
\usepackage{makecell}  % Required for specifying multiple rows within a table cell
\usepackage[table]{xcolor}  % Required for alternating line colours of tables

%-- Setting for headers ---%
\usepackage{fancyhdr}
\pagestyle{fancy}
\renewcommand{\headrulewidth}{0pt}  % Removes line under header
\lhead{\includegraphics[width=0.6\textwidth]{figures/logo.png}}

%-- Packages for Proper Handling of Figures --%
\usepackage{graphicx}
\usepackage{float}


%---------------------------------------------------------------------------------------
\begin{document}

%-- Without the *, vspace is cleared at the beginning of each page --%
\vspace*{0.01cm}

%-----------------------------------%
%  Course Info                         %
%-----------------------------------%
\begin{center}
	\Large{\textbf{Molecular and Cell Biology  - BIOL 1201\\
	Fall 2019\\
	Course Outline}}
\end{center}


%-----------------------------------%
%      Instructor Info              %
%-----------------------------------%
\textbf{\underline{Instructor Information}:}\\
\\
	%-- The @{} removes the spacing before that column, aligning these to the left (otherwise they are indented a bit) --%
	\begin{tabular}{@{} p{2.3cm} l }
		\textbf{Instructor:}	& Dr. Timothy R. Frasier (``Tim''), Associate Professor\\
		\\
		\textbf{Contact:} 	& Office: S327\\
					& E-mail: timothy.frasier@smu.ca\\
					& Tel: 902-491-6382\\
		\\
		\textbf{Office Hours:} & T 1:00--4:00\\
			& W 9:30--12:30 \\
			& or by appointment				
	\end{tabular}

	\begin{center}
		\rule{4cm}{0.5pt}
	\end{center}


%-----------------------------------%
%      Course Info                  %
%-----------------------------------%
\textbf{\underline{Course Information}:}


	%----------------------------------%
	%  Course times and locations      %
	%----------------------------------%
	\begin{tabular}{@{} p{2.3cm} l l l}
		\textbf{Lectures:} & TR & 11:30--12:45 & MM Theatre\\
		\addlinespace
		\textbf{Wet Labs:} 	& M, T, W, R & 2:30--5:30 & S 139\\
							& F & 9:30--12:29 & S 139\\
		\addlinespace
		\textbf{Tutorials:}	& M, T, W, R & 4:00--5:15 & LA 179\\
							& F & 10:00--11:15 & LA 176\\
		\addlinespace
			& \multicolumn{3}{p{14cm}}{\emph{Each lab section is divided into two groups: alpha and beta. One week the alpha group will do the wet labs and the beta group will do the tutorial. This will alternate the following week, and follow this alternating pattern throughout the term. Thus, each student will attend the wet labs every-other week, and the tutorials in the other every-other weeks.}}\\
		\addlinespace						
			& \multicolumn{3}{p{14cm}}{\emph{Note that the labs are run independently of the lectures for this course. I am teaching the lecture component, whereas Dr. Jessica Boyd (jessica.boyd@smu.ca) is coordinating the laboratory component. Therefore, if you have questions about the labs, please contact your TA or Dr. Boyd.}}
	\end{tabular}
	
	
	%----------------------------------%
	%  Course Sites                    %
	%----------------------------------%
	\begin{tabular}{@{} p{2.3cm} p{13.9cm}}
		\textbf{Course Sites:} & The course syllabus, as well as lecture materials, will be placed on the course Brightspace page. You should also see Brightspace for detailed reading assignments for the labs. These should be read prior to attending the lab, so that you will understand what's happening in the lab, learn more, and enjoy the labs more.\\
	\end{tabular}


	%----------------------------------%
	%  Course Description              %
	%----------------------------------%
	\textbf{Course Description:}\\
	\begin{tabular}{@{} p{2.3cm} p{13.9cm}}
		 & This course is an introduction to the principles of life at the cellular and molecular level. Major topics include the molecules that encode biological information, prokaryotic and eukaryotic cells, cell membranes and compartments, cellular respiration, photosynthesis, transcription and translation, DNA replication and cell division, mutation, variation and inheritance.
	\end{tabular}

	%------------%
	% Credits    %
	%------------%
	\textbf{3 credit hours}

	%----------------------------------%
	%  Pre-requisites                  %
	%----------------------------------%
	\begin{tabular}{@{} p{2.6cm} l }
		\textbf{Pre-requisites:} & Nova Scotia grade 12 biology or equivalent.\\ 
	\end{tabular}


	%----------------------------------%
	%  Required Materials             %
	%----------------------------------%
	\textbf{Required Materials:}
	\begin{enumerate}[topsep=-8pt]
		\item \textbf{Textbook \#1:} \emph{The Story of Life: Great Discoveries in Biology} by Sean B. Carroll. W.W. Norton.
		\item \textbf{Textbook \#2:} \emph{Campbell's Biology} Second Canadian Edition by Reece \emph{et al.} Pearson.\\
				Including access to the eBook, Mastering Materials, and Learning Catalytics
		\item \textbf{Lab Coat \& Safety Glasses:} Are available for purchase at the University bookstore. You must bring your lab coat and safety glasses to every lab.
		\item \textbf{WHMIS Training:} Must be completed by \textcolor{red}{Sep. 30}		
		\item \textbf{Calculator:} You will not likely need a calculator for this course (although there will be some basic math!). However, if you feel like you need one, the only approved calculators for tests in Biology, Chemistry, and Physics are the Casio fX260 or fX300.\\
	\end{enumerate}	


	%-----------------------------------%
	%      Learning Outcomes            %
	%-----------------------------------%
	\textbf{Learning Outcomes:}\\
	By the end of this course, students should be able to:
		\begin{itemize}[topsep=-8pt]
			\item Explain the scientific process and how it leads to a better understanding of the world.
			\item Describe the basic structure of cells, and explain the different ways in which they can communicate with one another, and the ways in which they can control what enters and leaves their interior.
			\item Describe the structure of DNA and explain how this structure leads to its functional properties as the hereditary material.
			\item Describe each of mitosis and meiosis and explain how they are similar and how they differ.
			\item Explain how sexual reproduction, recombination, and mutations represent important sources of variation, and describe the importance of each.
			\item Describe the structure of genes, and how this structure - along with the processes of transcription and translation -  result in the proteins needed for life.
			\item Explain the basic principles of Mendelian inheritance, and calculate expected offspring genotype and phenotype proportions given parental information.
			\item Describe how evolution can shape allele frequencies within a population, and how this process can be reconstructed using phylogenetic analyses.\\
		\end{itemize}
		

\newpage		
%------------------------------------%
% Course Content & Schedule          %
%------------------------------------%
\textbf{Course Content and Planned/Tentative Schedule:}\\
\emph{The schedule below is TENTATIVE. We will try to stick to this schedule, but it is likely that we will get off-track at one point or another. Necessary changes to the schedule will be made accordingly, and you will be notified of any changes during class hours.}	

	\begin{table}[H]
		%\footnotesize
		\centering
		\rowcolors{2}{white}{black!15}
		\begin{tabular}{l p{6cm} p{6cm}}
			\toprule
			\textbf{Day} & \textbf{Topic} & \textbf{Relevant Readings}\\
			\midrule
			Sep. 5 & Introduction & Campbell p. 1--16\\
			\addlinespace
			Sep. 10 & Science \& the scientific method & \makecell[tl]{Carroll Ch. 1 \& 2\\ Campbell p. 17--26}\\
			\addlinespace
			Sep. 12 & Evolution & \makecell[tl]{Carroll Ch. 7\\ Campbell p. 11--16, 492--503}\\
			\addlinespace
			Sep. 17 & Cell structure & \makecell[tl]{Carrol Ch. 9\\ Campbell p. 104--115, Ch. 7}\\
			\addlinespace
			Sep. 19 & Cell communication & Campbell Ch. 11\\
			\addlinespace
			Sep. 24 & Case study \#1 - MHC & None\\
			\addlinespace
			Sep. 26 & DNA as the hereditary material & \makecell[tl]{Carroll Ch. 3 \& 4\\ Campbell p. 329--334}\\
			\addlinespace
			Oct. 1 & \textbf{Midterm Exam \#1} & None \\ 
			\addlinespace
			Oct. 3 & DNA structure \& chemical nature & Carroll Ch. 5\\
			\addlinespace
			Oct. 8 & Transcription \& translation & Campbell Ch. 17\\
			\addlinespace
			Oct. 10 & Protein structure, folding, \& interaction & Campbell Ch. 5\\
			\addlinespace
			Oct. 15 & Chromosome structure \& mitosis & Campbell Ch. 12\\
			\addlinespace
			Oct. 17 & Meiosis \& sexual reproduction & Campbell Ch. 13\\
			\addlinespace
			Oct. 22 & Sources of variation: recombination & Campbell 274--279, 314--321\\
			\addlinespace
			Oct. 24 & Sources of variation: mutations & Campbell 372--377\\
			\addlinespace
			Oct. 29 & \textbf{Midterm Exam \#2} & None\\ 
			\addlinespace
			Oct. 31 & Mendelian inheritance & Campbell Ch. 14\\
			\addlinespace
			Nov. 5 & Mendelian inheritance (continued) & Campbell Ch. 14\\
			\addlinespace
			Nov. 7 & Allele frequencies, selection, \& patterns & None\\
			\addlinespace
			Nov. 12 & NO CLASS - FALL BREAK & None\\
			\addlinespace
			Nov. 14 & NO CLASS - FALL BREAK & None\\
			\addlinespace
			Nov. 19 & Case study \#2: antibodies \& gene structure & \makecell[tl]{Carroll Ch. 18\\ Campbell p. 358--363}\\
			\addlinespace
			Nov. 21 & Evolution of life & Campbell 550--569\\
			\addlinespace
			Nov. 26 & Phyogenetics I & \makecell[tl]{Carroll Ch. 10\\ Campbell Ch. 28}\\
			\addlinespace
			Nov. 28 & Phyogenetics II &\\
			\addlinespace
			Dec. 3 & TBA &\\
			\bottomrule
		\end{tabular}
	\end{table}	


	\newpage
	%------------------------------------%
	% Methods of Course Delivery %
	%------------------------------------%
	\textbf{Methods of Course Delivery:}\\
	This course has three components of delivery: the textbooks, lectures, and the laboratory. The \emph{textbooks} are the foundation of information in this course and should be read ahead of the appropriate classes so that you are ready for the information and so that the lecture can complement what you have learned and/or read from the book. The \emph{lecture} is not just an explanation of the text. Instead, I will use lecture time to try to explain more thoroughly, or delve deeper into, the topics from the relevant chapters that I think are most important. In this way, lecture is not a replacement for the textbooks, but rather should be complimentary to them. This also means that you need to both read the textbooks and take notes in lectures to do well on the exams. The \emph{laboratory} component provides an opportunity for you to gain hands-on experience with some of the key ideas from the course, to learn how to plan, conduct, and write-up experiments, and to become familiar practicing the scientific method.
	 

\vspace{0.3cm}


	%-----------------------------------%
	%      Marking Scheme               %
	%-----------------------------------%
	\textbf{Marking Scheme:}
		\begin{table}[H]
		\centering
			\begin{tabular}{p{2.5cm} r p{10cm}}
				\toprule
				\textbf{Component} & \textbf{\% of Final Grade} & \textbf{Description}\\
				\midrule
				Labs & 30\% & This will be composed of a variety of assignments, which you will learn more about during the laboratory and tutorial session. Note that \emph{Your final mark in the laboratory component must be greater than 50\% to gain credit for BIOL 1201}.\\
				\addlinespace
				In-class clickers & 5\% & During class on most days there will be one or more in-class ``quizzes'' that you will answer via your smart phone. These are used more as a tool to ensure everyone is on the same page, rather than as a ``test'' of your knowledge. Therefore, for each of these 50\% of your score will just be based on participation, and the other 50\% will be based on if you answered correctly or not.\\
				\addlinespace
				Exam \#1 & 20\% & This exam will be held during class time on Oct. 1. The objective is to test your understanding of the material up until this stage.\\
				\addlinespace
				Exam \#2 & 20\% & This exam will be held during class time on Oct. 31. The objective is to test your understanding of the material up until this stage.\\
				\addlinespace
				Exam \#3 (Final Exam) & 25\% & Your final exam will be held during the formal exam period and is scheduled by the Registrar. Please see the University Special Exams policy (Academic Regulation \#10, p. 29 in the \href{https://smu.ca/webfiles/AcademicCalendar2019-2020-Undergraduate.pdf}{\textcolor{blue}{Academic Calendar}}) for further information. \\
				\midrule
				& \textbf{100\%}\\
				\bottomrule
			\end{tabular}
		\end{table}	
		
		

	\newpage
	%----------------------------------------%
	%  Description of Course Components      %
	%----------------------------------------%
	\textbf{Description of Course Components:}
		\begin{table}[H]
			\begin{tabular}{@{} p{3.0cm} p{13.2cm}}
				\textbf{Labs} & The details regarding the grade breakdown of the laboratory component of the course (30\%) will be provided later, along with other laboratory materials and information.\\
				\addlinespace
				\textbf{In-class Clickers} & In most of the lecture periods there will be in-class ``quizzes'' that you will answer using your cell phones and the Learning Catalytics software associated with the textbook. These are meant just as frequent checks to make sure we are all on the same page, and that major concepts have been grasped, as opposed to being official quizzes. Therefore, they are only worth a small percentage of your grade (5\%): 50\% of which will just be based on whether or not you attempted to answer the questions, and the other 50\% will be based on whether or not you got the correct answer.\\
				\addlinespace
				\textbf{Mid-term Exams} & There will be two mid-term exams (1.25 hours each), spaced at approximately one-third and two-thirds of the way through the term. One will take place on Oct. 1 and the second will be on Oct. 29. Both of these will occur \textbf{during regular lecture hours}. The tests will not be cumulative. All questions will be multiple-choice, but you must know the course material well to succeed on the test. You will need your Saint Mary's University student identification card, a pencil, and an eraser for the multiple-choice computer answer sheet.\\
				\addlinespace
				\textbf{Final Exam} & The final exam will be 1.25 hours long, will not be cumulative, and will be scheduled during the final exam period in December. All of the questions will be multiple-choice, but you must know the course material well to succeed on the test. You will need your Saint Mary's University student identification card, a pencil, and an eraser for the multiple-choice computer answer sheet.\\
			\end{tabular}
		\end{table}


	\vspace{0.3cm}
	
	%---------------------------------------------------------------------%
	%  Student Responsibilities, Academic Integrity, & Code of Conduct    %
	%---------------------------------------------------------------------%
	\textbf{Student Responsibilities, Academic Integrity, \& Code of Conduct}
		\begin{enumerate}[topsep=0pt]
			\item Treat your colleagues and instructors with respect and give others your attention when they are speaking.
			\item \textbf{Technology in the Classroom:}
				\begin{enumerate}
					\item Smart phones may be used to take notes, but all the sounds must be turned off and you should not receive/send calls/texts, or check email during class.
					\item Please do not record lectures without my direct approval.
				\end{enumerate}
			\item \textbf{Academic integrity: As in all courses, plagiarism and cheating will not be tolerated.} You must hand in your own work, written in your own words. Plagiarism will be dealt with according to policies outlined in the Academic Calendar. It is your responsibility to familiarize yourself with Saint Mary's policies on Academic Integrity by consulting the ``\emph{Academic Integrity and Student Responsibility}'' (p. 19--27) and ``\emph{Academic Regulations}'' (p. 28--42) sections of the \href{https://smu.ca/webfiles/AcademicCalendar2019-2020-Undergraduate.pdf}{\textcolor{blue}{Academic Calendar}}, in order to be well informed on the consequences of dishonest behaviour.  
		\end{enumerate}

	\vspace{0.3cm}
 

	%-------------------------%
	%  Missed Classes         %
	%-------------------------%
	\textbf{Missed Classes:}\\
	You must attend every laboratory session. However, whether or not you attend lecture is up to you. SMU faculty no longer accept ``sick notes'' for missed days of class or exams. Instead, if students miss a day of class, particularly when there was something due that day (e.g., a research presentation or mid-term exam), they need to read, print out, fill out, and sign a copy of the \href{http://www.smu.ca/webfiles/Declaration_of_Extenuating_Circumstances_withinTerm.pdf}{\textcolor{blue}{Declaration of Extenuating Circumstances}}. This should then be submitted to the professor, and they will keep a copy, and also give a copy to the Science Advising Centre for your records. Students who miss the mid-term exam need to follow the instructions above, and then make an appointment to re-take the exam. There are pre-set times for this, which can be found by searching for ``missed exam dates'' on the SMU website. Arrange this with me (the professor), and I will ensure that the correct materials are in the right place at the right time. If a student misses the final exam, then the university follows Academic Regulation \#10 from the \href{https://smu.ca/webfiles/AcademicCalendar2019-2020-Undergraduate.pdf}{\textcolor{blue}{Academic Calendar}} (p. 29). For this, students do not interact with the professor. Instead, they consult with the Science Advising Centre, who then contacts the professor to develop a solution.
	
	\textbf{Missed Labs:}\\
	If you miss a lab, send your ``Declaration'' form together with supporting documentation to Dr. Jessica Boyd by email: jessica.boyd@smu.ca. When the lab schedule allows and space is available, we will try to fit you into another lab during the same week. Due to the complexity of the teaching lab schedule, labs can never be done on weeks other than scheduled. If rescheduling to another day in the same week isn't possible, we'll have to decide between a grade of zero or a grade transfer onto the remaining labs. This decision will be made on a case-by-case basis. Note that in no case will you be allowed to miss more than one lab without penalty.


	\vspace{0.3cm}
	
	%----------------------------------------------%
	%  Accessibility                               %
	%----------------------------------------------%
	\textbf{Accessibility:}\\
	The Fred Smithers Centre establishes individualized support services to help students with physical, medical, and learning disabilities. Accommodations work best for all concerned if the student comes forward to the Smithers Centre early.  Students are encouraged to seek more information by visiting the Centre, or its \href{http://www.smu.ca/campus-life/fred-smithers-centre.html}{\textcolor{blue}{website}}.	 
\end{document}